A primeira linha da entrada contém um inteiro $T$ ($1 \le T \le 10^4$) indicando o número de casos de testes.

A primeira linha de cada caso de teste contém quatro inteiros $x_L, y_L, x_A, y_A$ ($-500 \le x_L, y_L, x_A, y_A \le 500$), representando as coordenadas de Leonardo e Alberto, respectivamente.

A segunda linha contém outros quatro inteiros $x_{E1}$, $y_{E1}$, $x_{E2}$, $y_{E2}$ ($-500 \le x_{E1}, y_{E1}, x_{E2}, y_{E2} \le 500$), representando as coordenadas dos extremos do espelho.

É garantido que nenhum dos quatro pontos $L$, $A$, $E1$ e $E2$ ocupam a mesma posição no espaço.