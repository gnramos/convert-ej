% Exemplo de uso da classe UnBExam.
%
% Criado por : Guilherme N. Ramos (gnramos@unb.br) em 2011.

\documentclass{UnBExam}%

\disciplina{Estruturas de Dados}%
\turma{B}%
\documento{Prova 1}%
\professor{Guilherme N. Ramos}%

% \printanswers%

\begin{document}%
	\begin{questions}%
		\question[1] Questão 1, valendo 1 ponto.
			\begin{resposta}[Resposta]%
				Resposta da questão 1.
			\end{resposta}%
		\question[3] Questão 2, valendo 3 pontos.
			\begin{resposta}%
				Resposta da questão 3.
			\end{resposta}%
		\question[6] Questão 3, valendo 6 pontos.
			\begin{resposta}%
				\begin{parts}%
					\part[5] Primeira parte da resposta da questão 6.
					\part[1] Segunda parte.
				\end{parts}%
			\end{resposta}%
	\end{questions}%
\end{document}%
