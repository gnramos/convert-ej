\begin{problem}{Atendimento}{standard input}{standard output}{1 second}{256 megabytes}

Um supermercado tem $N$ funcionários que trabalham no caixa, identificados por números de $1$ a $N$. Assume-se que cada funcionário gasta $v_i$ segundos para processar um item de compra. Ou seja, se um cliente tem $c_j$ itens em sua cesta, um caixa levará $v_i*c_j$ segundos para processar todos os itens deste cliente.

Ao entrar na fila do caixa para ser atendido, um cliente aguarda ser atendido por um caixa livre. Se mais de um caixa estiver livre ao mesmo tempo, o cliente será atendido pelo caixa que possuir o menor número de identificação. Tal caixa só estará livre novamente após processar todos os itens deste cliente.

Há $M$ clientes na fila para serem atendidos, cada um com um determinado número de itens na sua cesta. Dadas as informações sobre os funcionários nos caixas e os clientes, o gerente pediu sua ajuda para descobrir quanto tempo levará para atender todos os clientes da fila do supermercado.

\InputFile
A primeira linha conterá dois inteiros $N$ e $M$, indicando o número de funcionários no caixa e o número de clientes, respectivamente $(1 \leq N \leq M \leq 10^4)$.

A segunda linha descreve $N$ inteiros $v_i$, indicando quanto tempo leva para o $i$-ésimo funcionário processar um item $(1 \leq v_i \leq 100$, para todo $1 \leq i \leq N)$.

A terceira linha apresenta $M$ inteiros $c_j$, indicando quantos itens o $j$-ésimo cliente tem em sua cesta $(1 \leq c_j \leq 100$, para todo $1 \leq j \leq M)$.

\OutputFile
Imprima uma linha contendo um valor inteiro, que expressa o tempo total gasto para que todos os clientes sejam atendidos.

\Examples

\begin{example}
\exmpfile{example.01}{example.01.a}%
\exmpfile{example.02}{example.02.a}%
\exmpfile{example.03}{example.03.a}%
\end{example}

\end{problem}

