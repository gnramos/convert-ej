Um supermercado tem $N$ funcionários que trabalham no caixa, identificados por números de $1$ a $N$. Assume-se que cada funcionário gasta $v_i$ segundos para processar um item de compra. Ou seja, se um cliente tem $c_j$ itens em sua cesta, um caixa levará $v_i*c_j$ segundos para processar todos os itens deste cliente.

Ao entrar na fila do caixa para ser atendido, um cliente aguarda ser atendido por um caixa livre. Se mais de um caixa estiver livre ao mesmo tempo, o cliente será atendido pelo caixa que possuir o menor número de identificação. Tal caixa só estará livre novamente após processar todos os itens deste cliente.

Há $M$ clientes na fila para serem atendidos, cada um com um determinado número de itens na sua cesta. Dadas as informações sobre os funcionários nos caixas e os clientes, o gerente pediu sua ajuda para descobrir quanto tempo levará para atender todos os clientes da fila do supermercado.
