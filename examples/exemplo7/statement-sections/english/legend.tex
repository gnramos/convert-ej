Tubertino é um sujeito aventureiro e um exímio caçador de tesouros em ambientes hostis. Por isso, ele vive realizando trilhas em lugares perigosos atrás de pedras preciosas para se tornar cada vez mais rico. Certo dia, Tubertino almoçou em um restaurante em Alto Paraíso de Goiás e ouviu de uma senhora que alienígenas deixaram uma grande quantidade de pedras preciosas em uma caverna, mas que nenhum aventureiro decidiu explorá-la devido ao risco de existirem alienígenas vigiando a localidade. 

Tubertino decidiu ir até essa caverna e após uma caminhada de 4 dias, encontrou um enorme baú com diversos objetos em seu interior. Em meio à escuridão da caverna, Tubertino pôde identificar sacos amarrados de diversos tamanhos e outros objetos espalhados que aparentemente não possuíam forma específica de qualquer pedra preciosa conhecida. 

Elabore um programa que, dada a configuração dos objetos do baú, determine a quantidade de pedras preciosas que Tubertino consegue identificar, isto é, a quantidade de objetos que estão no interior dos sacos. Repare que alguns sacos podem estar incluídos em outros sacos e que podem haver pedaços de sacos rasgados e espalhados pelo baú.
