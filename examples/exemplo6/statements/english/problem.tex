\begin{problem}{Jogo familiar}{standard input}{standard output}{1 second}{256 megabytes}

Valdete e Drauzio elaboraram um jogo para entreter seus familiares no Natal. A ideia do jogo consiste em enfileirar todos os $N$ familiares e sortear $M$ números, que indicarão várias modificações na fila formada. Se um número sorteado $s$ for positivo e maior do que zero, são realizadas $s$ operações nessa fila, em que o familiar no início da fila é posicionado no final dela (um familiar por vez sai do início da fila e se desloca para o último lugar da fila). Caso o valor $s$ seja igual a -1, o familiar na primeira posição da fila é excluído do jogo. O vencedor do jogo será o familiar que, após finalizar o sorteio dos $M$ elementos, estiver na primeira posição da fila.

Elabore um programa em linguagem C que determine o familiar de Valdete e Drauzio que vencerá o o jogo.


\InputFile
A primeira linha da entrada consiste em dois inteiros $2 \leq N \leq 10^4$ e $1 \leq M < N$. A próxima linha apresenta $M$ valores $s_1, s_2, ... , s_M$, em que cada $-N \leq s_i < N$ se refere ao i-ésimo número sorteado. Considere que cada familiar é representado por um número inteiro entre $1$ e $N$.


\OutputFile
Imprima o número inteiro associado ao familiar que vencerá o jogo proposto por Valdete e Drauzio.

\Examples

\begin{example}
\exmpfile{example.01}{example.01.a}%
\exmpfile{example.02}{example.02.a}%
\exmpfile{example.03}{example.03.a}%
\end{example}

\Note
*** No primeiro caso de teste, os familiares são inicialmente enumerados em:

1 -> 2 -> 3 -> 4 -> 5 -> 6

O primeiro número sorteado (2) indica que os familiares 1 e 2 devem ir para o final da fila, um por vez, resultando em:

3 -> 4 -> 5 -> 6 -> 1 -> 2

O segundo número sorteado (2) faz com que os familiares 3 e 4 vão para o final da fila, um por vez, resultando em:

5 -> 6 -> 1 -> 2 -> 3 -> 4 

Logo, o familiar número 5 foi o vencedor do jogo.

*** No segundo caso de teste, os familiares também são enumerados e enfileirados em:

1 -> 2 -> 3 -> 4 -> 5 -> 6

O primeiro número sorteado (3) indica que os familiares 1, 2 e 3 devem ir para o final da fila, um por vez, resultando em:

4 -> 5 -> 6 -> 1 -> 2 -> 3 

O segundo número sorteado (-1) faz com que o familiar 4 seja excluído do jogo. A fila fica como:

5 -> 6 -> 1 -> 2 -> 3 

O terceiro número sorteado (-1) faz com que o familiar 5 seja excluído do jogo. A fila fica como:

6 -> 1 -> 2 -> 3

O terceiro número sorteado (-1) faz com que o familiar 6 seja excluído do jogo. A fila fica como:

1 -> 2 -> 3

Logo, o familiar número 1 foi o vencedor do jogo.

\end{problem}

